% -*- coding: utf-8 -*-
% Plantilla básica de tesis en español
\documentclass[12pt,oneside,a4paper]{report}

% Paquetes
\usepackage[spanish]{babel}
\usepackage[utf8]{inputenc}
\usepackage[T1]{fontenc}
\usepackage{lmodern}
\usepackage{microtype}
\usepackage{csquotes}
\usepackage{setspace}
\usepackage[a4paper,margin=3cm]{geometry}
\usepackage{graphicx}
\usepackage{float}
\usepackage{booktabs}
\usepackage{amsmath,amssymb}
\usepackage{enumitem}
\usepackage{caption}
\usepackage{subcaption}
\usepackage{tocbibind}
\usepackage{xcolor}
\usepackage{hyperref}
\usepackage{cleveref}
\usepackage[backend=biber,style=authoryear,sorting=nyt]{biblatex}

\addbibresource{bibliografia.bib}

\graphicspath{{figuras/}}

\hypersetup{
  pdftitle={Título de la tesis},
  pdfauthor={Nombre Apellido},
  pdfsubject={Tesis de Grado},
  pdfkeywords={neuralshield, seguridad, aprendizaje automático},
  colorlinks=true,
  linkcolor=blue,
  citecolor=teal,
  urlcolor=blue
}

% Configuración general
\setlength{\parindent}{15pt}
\setlength{\parskip}{0.5em}

\begin{document}

% Portada
\begin{titlepage}
  \centering
  {\Large Nombre de la Universidad\par}
  {\large Facultad o Departamento\par}
  \vspace{2cm}
  {\Huge\bfseries Título de la Tesis\par}
  \vspace{0.5cm}
  {\Large Subtítulo opcional\par}
  \vspace{2cm}
  {\large Autor: Nombre Apellidos\par}
  {\large Director: Nombre Apellidos\par}
  \vfill
  {\large Ciudad, País\par}
  {\large Mes YYYY\par}
\end{titlepage}

\pagenumbering{roman}

% Dedicatoria (opcional)
% \chapter*{Dedicatoria}
% \addcontentsline{toc}{chapter}{Dedicatoria}
% Texto de dedicatoria...

% Agradecimientos (opcional)
\chapter*{Agradecimientos}
\addcontentsline{toc}{chapter}{Agradecimientos}
Escriba aquí sus agradecimientos.

% Resumen
\chapter*{Resumen}
\addcontentsline{toc}{chapter}{Resumen}
Breve resumen de la tesis en español (150–300 palabras).

\textbf{Palabras clave:} lista, de, palabras, clave.

% Abstract (opcional)
% \chapter*{Abstract}
% \addcontentsline{toc}{chapter}{Abstract}
% Short abstract in English (150–300 words).
% \textbf{Keywords:} list, of, keywords.

\tableofcontents
\listoffigures
\listoftables

\cleardoublepage
\pagenumbering{arabic}
\onehalfspacing

\chapter{Introducción}
Contexto, motivación, objetivos y contribuciones.
% Ejemplo de figura (elimine los % cuando tenga el archivo)
% \begin{figure}[H]
%   \centering
%   \includegraphics[width=0.7\textwidth]{figura-ejemplo}
%   \caption{Ejemplo de figura.}
%   \label{fig:ejemplo}
% \end{figure}

\chapter{Marco teórico}
Conceptos, trabajos relacionados y fundamentos.

\chapter{Metodología}
Diseño experimental, datos y procedimientos.

\chapter{Implementación}
Detalles de implementación del sistema \emph{NeuralShield} y componentes.

\section{Preprocesamiento}
\subsection{Especificaciones}

\subsubsection{Normalizar y agregar flags de rarezas}
El primer paso es normalizar la entrada a un formato más conciso para el 'tokenizador' y 'encoder' de nuestro modelo. El output esperado para un mensaje HTTP genérico es el siguiente:
\begin{verbatim}
http://localhost:8080/tienda1/publico/autenticar.jsp?modo=entrar&login=chuang&pwd=visionario&remember=on&B1=Entrar
User-Agent: Mozilla/5.0 (compatible; Konqueror/3.5; Linux) KHTML/3.5.8 (like Gecko)
Pragma: no-cache
Cache-control: no-cache
Accept: text/xml,application/xml,application/xhtml+xml,text/html;q=0.9,text/plain;q=0.8,image/png,*/*;q=0.5
Accept-Encoding: x-gzip, x-deflate, gzip, deflate
Accept-Charset: utf-8, utf-8;q=0.5, *;q=0.5
Accept-Language: en
Host: localhost:8080
Cookie: JSESSIONID=29BDC1A4215FA40AC619614130C4A037
Connection: close
\end{verbatim}

\begin{verbatim}
M:GET
U:http://localhost:8080/tienda1/publico/autenticar.jsp?modo=entrar&login=chuang&pwd=<SECRET:alpha:10>&remember=on&B1=Entrar
H:accept=application/xhtml+xml;q1.0 application/xml;q1.0 image/png;q1.0 text/xml;q1.0 text/html;q0.9 text/plain;q0.8 */*;q0.5
H:accept-charset=utf-8;q1.0 *;q0.5
H:accept-encoding=deflate gzip x-deflate x-gzip
H:accept-language=en;q1.0
H:cache-control=no-cache
H:connection=close
H:cookie=JSESSIONID<len:32> [COOKIE:1]
H:host=localhost:8080
H:pragma=no-cache
H:user-agent=mozilla/5.0 konqueror/3.5 khtml/3.5.8 like:gecko
FLAGS:[HOPBYHOP:connection]
\end{verbatim}

Problema: La URL puede tener caracteres codificados dos veces, es decir, `%252E` = `%2E` codificado dos veces.
Solución: Decodificar la URL una sola vez y agregar el flag `DOUBLEPCT` cuando se detecte que hay caracteres codificados dos veces.
\subsubsection{Construir URL absoluta cuando es relativa}
Problema: La URL puede venir en forma particionada, es decir, con el path y el query separados.
Solución: Construir la URL absoluta canónica (`scheme://host[:port]/path?query`) cuando el request target viene en forma relativa (origin-form, p. ej., `GET /path HTTP/1.1`) y cómo tratar los casos de absolute-form y authority-form, así como discrepancias con el header `Host`.
\subsubsection{Doble Encoding}
Problema: La URL puede tener caracteres codificados dos veces, es decir, `%252E` = `%2E` codificado dos veces.
Solución: Decodificar la URL una sola vez y agregar el flag `DOUBLEPCT` cuando se detecte que hay caracteres codificados dos veces.



\subsection{Implementación}
\subsubsection{Pipeline}


\section{Extracción de \textit{features}}














































\chapter{Resultados y discusión}
Presentación, análisis y discusión de resultados.

\chapter{Conclusiones y trabajo futuro}
Conclusiones principales, limitaciones y líneas de trabajo futuro.

\appendix
\chapter{Apéndice: material complementario}
Detalles adicionales, pruebas complementarias, listados de código, etc.

\printbibliography

\end{document}

